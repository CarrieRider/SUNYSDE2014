%\documentclass[12pt,letterpaper]{amsart}
%\setlength{\oddsidemargin}{.0in}
%\setlength{\evensidemargin}{.0in}
%\setlength{\textwidth}{6.5in}
%\setlength{\topmargin}{-.3in}
%\setlength{\headsep}{.20in}
%\setlength{\textheight}{9.in}
%\usepackage[leqno]{amsmath}
%\usepackage{amsfonts}
%\usepackage{amssymb}
%\usepackage{amsthm}
%\usepackage{amssymb}
%\usepackage[all]{xy}
%\usepackage{graphicx}



%Here are some user-defined notations
\newcommand{\RR}{\mathbf R}
\newcommand{\CC}{\mathbf C}
\newcommand{\ZZ}{\mathbf Z}
\newcommand{\ZZn}[1]{\ZZ/{#1}\ZZ}
\newcommand{\QQ}{\mathbf Q}
\newcommand{\rr}{\mathbb R}
\newcommand{\cc}{\mathbb C}
\newcommand{\zz}{\mathbb Z}
\newcommand{\zzn}[1]{\zz/{#1}\zz}
\newcommand{\qq}{\mathbb Q}
\newcommand{\calM}{\mathcal M}
\newcommand{\latex}{\LaTeX}
\newcommand{\tex}{\TeX}
\newcommand{\sm}{\setminus} 


%improving spacing in tables (space above and below characters in a row)
\newcommand{\tfix}{\rule{0pt}{2.6ex}}
\newcommand{\bfix}{\rule[-1.2ex]{0pt}{0pt}}



%Here are commands with variable inputs 
\newcommand{\intf}[1]{\int_a^b{#1}\,dx}
\newcommand{\intfb}[3]{\int_{#1}^{#2}{#3}\,dx}
\newcommand{\marginalfootnote}[1]{%
        \footnote{#1}
        \marginpar[\hfill{\sf\thefootnote}]{{\sf\thefootnote}}}
\newcommand{\edit}[1]{\marginalfootnote{#1}}


%Here are some user-defined operators
\newcommand{\Tr}{\operatorname {Tr}}
\newcommand{\GL}{\operatorname {GL}}
\newcommand{\SL}{\operatorname {SL}}
\newcommand{\Prob}{\operatorname {Prob}}
\newcommand{\re}{\operatorname {Re}}
\newcommand{\im}{\operatorname {Im}}


%These commands deal with theorem-like environments (i.e., italic)
%\theoremstyle{plain}
%\newtheorem{theorem}{Theorem}[section]
%\newtheorem{corollary}[theorem]{Corollary}
%\newtheorem{lemma}[theorem]{Lemma}
%\newtheorem{conjecture}[theorem]{Conjecture}

%These deal with definition-like environments (i.e., non-italic)
%\theoremstyle{definition}
%\newtheorem{definition}[theorem]{Definition}
%\newtheorem{example}[theorem]{Example}
%\newtheorem{remark}[theorem]{Remark}

%This numbers equations by section
\numberwithin{equation}{section}

\section{Brownian Motion}
\begin{frame}
  \frametitle{Outline}
  \tableofcontents[ currentsection ]
\end{frame}

\subsection{Random Walk}
%\begin{document}
\begin{frame}{Markov Process}
(\textbf{Markov Process}) is a stochastic process with the following properties: 
\begin{enumerate}
\item The number of possible outcomes or states is finite
\item The outcome at any stage depends only on the outcome of the previous stage.
\item The probabilities are constant over time.   
\end{enumerate}
\end{frame}


\begin{frame}{Brownian Motion}
\begin{definition}(\textbf{Brownian Motion}) is a stochastic process that models random continuous motion. The stochastic process $B=\{B(t), t\geq 0\}$ is standard Brownian Motion if the following holds:
\begin{enumerate}
\item $B$ has independent increments.
\item For $0 \leq s < t,$ $$B(t)-B(s) \sim N(0,t-s).$$
\item The paths of $B$ are continuous with probability $1$.
\item $B(0)=0$ 
\end{enumerate}
\end{definition}
\end{frame}

\begin{frame}
  %\frametitle{Without Noise}
  \movie[height=6.33cm,width=8.55cm,loop,poster]{Random Walk}{randomWalk.avi}

  \hyperlinkmovie[once]{randomWalk.avi}{Random Walk} 
  \cite{doi:10.1137/S0036144500378302}
\end{frame}


\begin{frame}{Scaling a Brownian Motion}
\vfill

If $x(t)$ is a Brownian Motion then 
$\displaystyle \frac{x(\lambda t)}{\sqrt{\lambda}}$ 
is a Brownian motion.
	
\vfill
\begin{eqnarray*}
  \lefteqn{
	P \left(a \leq \frac{x(\lambda t)}{\sqrt{\lambda}}-\frac{x(\lambda s)}{\sqrt{\lambda}} \leq b \right)
	} & & \\
    & = & P \left(a \sqrt{\lambda} \leq x(\lambda t) - x (\lambda s) \leq b \sqrt{\lambda} \right), \\
	& = & \displaystyle \frac{1}{\sqrt{2 \pi (\lambda t -\lambda s)}} 
                    \int_{a \sqrt{\lambda}}^{b \sqrt{\lambda}} e^{\frac{-x^2}{2}(\lambda t- \lambda s)} dx, \\
    & = & \frac{1}{\sqrt{2 \pi (t-s)}} \int_{a}^{b} e^{\frac{-u^2}{t-s}}du.
 \end{eqnarray*}
\end{frame}

\subsection{Integration}
\begin{frame}{Riemann-Stieltjes}
(\textbf{Riemann-Stieltjes Integral}): 
$$\displaystyle \int_{a}^{b} f(g)dg= \underset{n \to \infty}{\lim} \sum_{i=1}^{N} f(g(t_i)) \cdot (g(t_{i+1}))- g(t_i))$$\\

The main motivation for the Riemann-Stieltjes Integral comes from the concept of Cumulative Distribution Function (CDF) of a random variable. 
\end{frame}

\begin{frame}{Weiner Integral}
\textbf{Weiner Integral} $$\int_{a}^{b} f(t)dW(t)$$
$$1[t_{i+1}, t_i] (t)= \begin{cases} 1 & \text{if} \ t_{i+1} \leq t < t_i \\
0 & \text{otherwise} \end{cases}$$
\end{frame}


%\textbf{Multivariate Taylor Expansion:} $$F(t)-F(s)=F^{\prime}(s)(t-s)+\frac{1}{2}F^{''} (s)(t-s)^2 + \frac{1}{3!}F^{'''}(t-s)^3+ \text{H.O.T}$$
\subsection{It\^o's Formula}

\begin{frame}
\frametitle{It\^o's Formula}
It\^o's Formula is used in It\^o Calculus to find the differential of a time-dependent function of a stochastic process.
\vfill

		 \begin{block}{Differential Form}
      \begin{align*}
				\displaystyle \partial x_t =\left(\frac{\partial x}{\partial t} + \frac{1}{2} \left(\frac{\partial ^2 x}{\partial B ^2}\right)\right) \cdot  dt + \frac{\partial x}{\partial B} \cdot dB 
			\end{align*}
    \end{block}
		
\vfill

		\begin{block}{Integral Form}
      \begin{align*}
				\displaystyle F(t, B(t))-F(a,B(a))=
 \int_{a}^{t} \frac{\partial F}{\partial s} + \frac{1}{2} \frac{\partial^2 F}{\partial B^2}ds+
 \int_a^t \frac{\partial F}{\partial B} dB
			\end{align*}
    \end{block}

\vfill
\end{frame}

\begin{frame}{It\^o's Formula} 

Let $F=tB^2$ 
\begin{align*}
\displaystyle
&\frac{dF}{dt}= B^2\\
&\frac{dF}{dB}=2t+B\\ 
&\frac{d^2 F}{dB^2}=2t
\end{align*}
\begin{align*}
tB^2(t)-aB^2(a) &=\int_{a}^{t}B^s ds+ \int_{a}^{t}2sBdB+\int_a^t \frac{1}{2}2sds\\
 &=\int_a^b B^2+sds+\int_a^t 2sBdB\\
 &\text{OR}\\
tB^2(t)-aB^2(a) &= \int_a^t B^s ds+ \int_a^t 2sBdBt+ \frac{1}{2} t^2- \frac{1}{2}a^2
\end{align*}
\end{frame}

%\section{Nondimensionalization}
%\begin{frame}
 % \frametitle{Outline}
  %\tableofcontents[ currentsection ]
%\end{frame}

%\begin{frame}{Nondimensionalization}
%\textbf{Nondimensionalization}: method to reduce parameters. 

%\begin{enumerate}
%\item
%List all the variables and parameters along with their dimensions.
%\item
%For each variable, say $x$, form a product (or quotient) $p$ of parameters that has the same dimensions as $x$, and define a new variable $y=\frac{x}{p}.$ $y$ is a "dimensionless" variable. It's numberical  value is the same no matter what system of units is used.
%\item Rewrite the differential equation in terms of the new variables.
%\item 
%In the new differential equation, group the parameters into nondimensional combinations, and define a new set of nondimensional parameters expressed as the nondimensional combinations of the original parameters. 
%\end{enumerate}
%\end{frame}


%\end{document}

